\documentclass{resume} % Use the custom resume.cls style

\usepackage[left=0.4 in,top=0.4in,right=0.4 in,bottom=0.4in]{geometry} % Document margins
\newcommand{\tab}[1]{\hspace{.2667\textwidth}\rlap{#1}} 
\newcommand{\itab}[1]{\hspace{0em}\rlap{#1}}

\name{Vimal Kumar} % Your name
\address{\bf Senior Software Engineer, Qualcomm}
% You can merge both of these into a single line, if you do not have a website.
\address{+91 xxxxxxxx \\ 
\href{mailto:xxxxxx}{xxxxxxx} \\ \href{https://www.linkedin.com/in/jayamvimal/}{LinkedIn}  }  %

\begin{document}

%----------------------------------------------------------------------------------------
%	OBJECTIVE
%----------------------------------------------------------------------------------------

\begin{rSection}{ABOUT}

Software Engineer with $\sim$8 years of experience in system software, camera pipelines, and Android middleware. 
Skilled in system design, performance optimization, and debugging, with hands-on experience delivering 
high-performance features for embedded devices. Strong background in leading technical teams 
and driving end-to-end solutions.


\end{rSection}



%----------------------------------------------------------------------------------------
%	EDUCATION SECTION
%----------------------------------------------------------------------------------------

\begin{rSection}{Education}

{\bf Master of Computer Science and Engineering}, National Institute of Technology, Warangal \hfill {2020 - 2022}\\
\smallskip \\
{\bf Bachelor of Information Technology}, Anna University, Chennai \href{https://www.youtube.com/watch?v=XCFW81O_ROs}{[UG Best Project Award - 2015]}\hfill {2012 - 2016}
%Minor in Linguistics \smallskip \\
%Member of Eta Kappa Nu \\
%Member of Upsilon Pi Epsilon \\
\end{rSection}


%----------------------------------------------------------------------------------------
% TECHNICAL STRENGTHS	
%----------------------------------------------------------------------------------------
\begin{rSection}{SKILLS}

\begin{tabular}{ @{} >{\bfseries}l @{\hspace{6ex}} l }

Languages & C++ 11/14/17 (Expert), Python, JavaScript, C\# (Intermediate) \\

System & Android HAL, Linux Internals, Multithreading, IPC, Memory Management, Full Stack  \\

Domains & Embedded Software, Camera Pipelines, XR/AR/VR Applications, Middleware, Perf Optimization \\

Tools & Git, ADB, GDB, Perf, Valgrind, OpenCV, TensorFlow, Android Studio, Jira \\

Soft Skills & Architecture Design, Code Review, Debugging, Technical Leadership, Cross-team Collaboration \\

\end{tabular}

\end{rSection}

\begin{rSection}{EXPERIENCE}

\textbf{Senior Software Engineer} \hfill July 2022 -- Present \\
Qualcomm \hfill \textit{Hyderabad}
\begin{itemize}
    \itemsep -3pt
    \item Designed and optimized Dynamic Foveated Camera Resolution Control on Halliday and Matrix chipsets, improving camera bandwidth utilization and system power efficiency.
    \item Developed, optimized, and debugged Android HAL and middleware components for XR smart-glass devices, working closely with camera, sensor, and display subsystems.
    \item Worked extensively on C++ system components, memory management, thread scheduling, and cross-process communication within the Android framework.
    \item Performed performance profiling, trace analysis, and low-level debugging using ADB, GDB, Perf, and vendor-specific tools.
    \item Supported camera bring-up, CSI/IPU pipeline validation, buffer management, and image quality tuning for early silicon.
    \item Built automation frameworks using Python and Shell scripting; contributed to CTS validation suite for XR camera and sensor features.
    \item Collaborated with SoC, firmware, and kernel teams for issue triage, root-cause analysis, and integration validation.
\end{itemize}

\textbf{Technical Lead Engineer} \hfill Oct 2016 -- Aug 2020 \\
TCS Research and Development (XR Lab) \hfill \textit{Chennai, Singapore}
\begin{itemize}
    \itemsep -3pt
    \item Led development of XR/AR/VR applications across smart-glass, mobile, and 6DOF VR headsets, ensuring high performance on constrained embedded hardware.
    \item Built C++ and Unity-based rendering pipelines, sensor fusion logic, and latency-optimized interaction flows.
    \item Worked on computer vision modules, integrating ML/TensorFlow models for gesture recognition and object detection.
    \item Delivered multiple PoCs for enterprise clients (airlines, healthcare) involving IoT integration, real-time data syncing, and low-latency XR interaction.
    \item Mentored junior engineers, reviewed system architecture, and owned feature delivery across mobile, web, and embedded stacks.
\end{itemize}

\end{rSection} 

%----------------------------------------------------------------------------------------
%	TOP PROJECTS (ATS-OPTIMIZED FOR HW/SW EMBEDDED, XR, COMPUTER VISION ROLES)
%----------------------------------------------------------------------------------------
\begin{rSection}{TOP PROJECTS}
\vspace{-1.25em}

\item \textbf{Low Frame-Rate Camera Optimization}  
{Implemented dynamic frame-rate control for embedded camera sensors using stream-off / power-off sequencing across request/result threads in the Android HAL. Reduced sensor power draw and system thermal load while preserving QoS for capture pipelines. [Power consumption reduced by 15%]}

\item \textbf{AI Copilot with Always Sensing Camera}  
{Architected and implemented an Always-On (AON) camera + on-device AI copilot for continuous image description and context-aware alerts. Optimized sensor sampling and inference scheduling to minimize power and CPU/GPU utilization on constrained SoC platforms.}

\item \textbf{Smartglass Security Pipeline}  
{Designed secure, low-latency sensor-to-application streaming on ARM-based smart-glass platforms. Implemented encrypted buffer transport, HAL-level access control, and secure IPC to prevent unauthorized buffer access while maintaining real-time throughput.

\item \textbf{Foveated Capture \& ISP-Level Cropping for XR}  
{Built foveation-driven capture that performs region-of-interest cropping in ISP/Sensor hardware to eliminate unnecessary frame transfers to GPU. Reduced memory-bandwidth usage and GPU load to extend battery life on XR headsets and smart glasses. [Reduced GPU and CPU overhead by directly foveating in ISP, Metrics depends on User's movement]}

\item \textbf{VHAB — Assistive XR Platform for Special-Needs Therapy}  
{Led development of a VR-based physiotherapy and behavioral-training platform integrating embedded VR clients, IoT sensors, and MERN analytics dashboard. Delivered computer-vision–based exercise tracking and automated progress analytics for deployments in therapy centers and homes. [Deployed in 5+ sites;  participation improved a lot]}

\end{itemize}
\end{rSection}

%----------------------------------------------------------------------------------------
%	AWARDS (CONCISE, HYPERLINKS, CONTEXT)
%----------------------------------------------------------------------------------------
\begin{rSection}{Awards} 
\begin{itemize}

\item \href{https://www.linkedin.com/posts/ayushi-manoria-094a83178_qualcomm-qbuzz2024-xr-activity-7333786063821242368-SKF9?utm_source=share&utm_medium=member_desktop&rcm=ACoAAB-NCXIBjdnoYSgckGOzHNbCtMl4SGQ7v0s}{\textbf{AIBuzz Award — 2024}}  
Awarded for an AI + XR virtual-assistant prototype that provides multimodal motion guidance for rehabilitation. Recognized for technical innovation and practical impact. [Confidential metrics available on request.]

\item \href{https://drive.google.com/file/d/1rXnLbqZIzRqTLbpFZkbQ1MeoSK2LpRcQ/view}{\textbf{On-the-Spot Award — 2017}}  
Granted by Singapore Airlines for delivery of a low-latency smart-glass solution used in operational trials.

\item \href{https://drive.google.com/file/d/1_Ffyj-XwmqnWTwdu9XUKIrCuG-ngw0wL/view}{\textbf{TMA CSR Award — 2018}}  
Recognized by the Governor of Kerala for VHAB — an XR + IoT healthcare initiative for special-needs children demonstrating social impact and technical execution.

\end{itemize}
\end{rSection}

%----------------------------------------------------------------------------------------
%	PUBLICATIONS & IP (CONFIDENTIALITY-PRESERVING)
%----------------------------------------------------------------------------------------
\begin{rSection}{Publications \& Intellectual Property} 
\begin{itemize}

\item \textbf{Journal Publication — XR for Remote MRO Assistance} \\ 
Published \textit{“Support for MRO Remote Assistance through Extended Reality (XR)”} in \textit{Computers \& Industrial Engineering} (CAIE-D-22-02027). Paper presents scalable XR workflows for maintenance, remote collaboration, and operational-efficiency improvements. [Add measurable outcomes: task-time ↓ XX\%]

\item \textbf{Invention Disclosure — XR + AI Intelligent Assistive System} \\ 
Filed an internal Invention Disclosure (IDF) at Qualcomm for an XR+AI assistive platform focused on real-time multimodal guidance and on-device inference. Details are confidential; high-level summary included to demonstrate IP activity and invention lifecycle experience.

\end{itemize}
\end{rSection}


%----------------------------------------------------------------------------------------
%\begin{rSection}{Leadership} 
%\begin{itemize}
%    \item Admin for the \href{https://discord.com/invite/WWbjEaZ}{FAANGPath Discord community} with over 6000+ job seekers and industry mentors. Actively involved in facilitating online events, career conversations, and more alongside other admins and a team of volunteer moderators! 
%\end{itemize}
%
%
%\end{rSection}


\end{document}
